% !TEX root = Clean-Thesis.tex
%
\pdfbookmark[0]{Acknowledgement}{Acknowledgement}
\chapter*{背景介绍}
\label{sec:acknowledgement}
\vspace*{-10mm}

电子自旋共振(ESR)或者电子顺磁共振(EPR)是指在含有未成对电子的原子、离子或者分子的顺磁性物质,在稳恒磁场的作用下对微波能量发生的共振吸收现象。如果共振仅仅涉及物质中的电子则被称为电子的自旋共振,一般情况下,电子的轨道磁矩不可忽略所以被称作电子的顺磁共振。

电子顺磁共振通过对于研究对象共振波谱的观测,可以了解这些物质中未成对电子的电子状态以及其周围环境方面的信息。同时由于这种方法并不改变或者破坏研究对象本身的性质,因而对于那些寿命短,化学性质不稳定的分子或者自由基等的研究有着显著的作用。所以自1945年发现这个现象以来,顺磁共振已经在物理学,化学等领域中取得了广泛的应用。

这个实验目的是利用微波系统和一系列的微波器件对DPPH自由基的电子自旋共振进行观测和测量,并与理论给出的数值进行验证。从实验的结果来看,实验数据基本上支持了理论给出的预测结果。
