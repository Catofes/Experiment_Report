\documentclass[slidestop, compress, red]{beamer}
\usepackage[fancyhdr,space,fntef,fontset=ubuntu]{ctex}
\usepackage{flushend,cuted}
\usepackage{amsfonts}
\usepackage{refcount}
\usepackage{textcomp}
\usepackage{feynmp-auto}
\usetheme{Copenhagen}
\usecolortheme{lily}

\begin{document}


%%-------------------------------------------------
    \title{Rosenbluth 公式}
    \author{马远卓 卢梦 张一镆 孙嘉琪 乔颢}
    \institute{北京大学物理学院}
    \date{\today}
    \frame{
    	\titlepage
    }
%%-------------------------------------------------

\frame{
	\section{思路}
	\frametitle{思路}
	\begin{block}{Rutherford}
		\begin{equation}
			|\frac{d\sigma}{d\Omega}|_R = \frac{4Z^2\alpha^2(\hbar c)^2 E'^2}{|qc|^4}
		\end{equation}
	\end{block}

	\begin{block}{Mott}
		\begin{equation}
			|\frac{d\sigma}{d\Omega}|_{Mott} = |\frac{d\sigma}{d\Omega}|_R(1-\beta\sin^2\frac{\theta}{2})
		\end{equation}
	\end{block}

	\begin{block}{Rosenbluth}
		\begin{equation}
			\frac{d\sigma}{d\Omega}=|\frac{d\sigma}{d\Omega}|_{Mott}(A \cos^2(\frac{\sigma}{2})-\frac{q^2}{2m}B\sin^2(\frac{\sigma}{2}))
		\end{equation}
	\end{block}
}

\section{经典卢瑟福散射}
\begin{frame}
	\frametitle{经典卢瑟福散射}
	\begin{block}{卢瑟福散射公式}
		\begin{equation}
			\frac{d\sigma}{d\Omega}(E, \theta) = \frac{(zZe)^2}{(4\pi\epsilon_0)^4(4E)^2\sin^4\frac{\theta}{2}}
		\end{equation}
	\end{block}
	要求:
	\begin{enumerate}
		\item 非相对论电子
		\item 不考虑核反冲
		\item 核为点粒子
		\item 弹性散射 
	\end{enumerate}
\end{frame}

\section{莫特截面}
\begin{frame}
	\frametitle{莫特截面}
	\begin{block}{考虑电子自旋}
		\begin{equation}
			|\frac{d\sigma}{d\Omega}|^*_{Mott} = |\frac{d\sigma}{d\Omega}|_R(1-\beta^2\sin^2\frac{\theta}{2})
		\end{equation}
	\end{block}
	\begin{block}{考虑核反冲}
		\begin{equation}
			|\frac{d\sigma}{d\Omega}|_{Mott} = |\frac{d\sigma}{d\Omega}|^*_{Mott}\frac{E'}{E}
		\end{equation}
	\end{block}
\end{frame}

\section{进一步的改进}
\subsection{加入核自旋}
\begin{frame}
	\frametitle{考虑核自旋,核子仍然视做点粒子}
	\begin{center}
	\begin{fmffile}{fmpicture}
	\begin{fmfgraph*}(140,70)
		\fmfleft{i1,i2}
		\fmfright{o1,o2}
		\fmf{fermion,label=$e^-,,P_1$}{i1,v1}
		\fmf{fermion,label=$e^-,,P_3$}{v1,i2}
		\fmf{fermion,label=$p,,P_2$}{o1,v2}
		\fmf{fermion,label=$p,,P_4$}{v2,o2}
		\fmf{photon,label=$\gamma,,q$}{v1,v2}
		\fmfdotn{v}{2}
	\end{fmfgraph*}
	\end{fmffile}
	\end{center}
	\begin{equation}
		\frac{d\sigma}{d\Omega} \sim |\mathcal{M}|^2
	\end{equation}
	\begin{equation}
		i\mathcal{M}=[\overline u (P_4)(-ie\gamma^\mu)u(P_2)]\frac{-ig_{\mu\nu}}{(P_1-P_3)^2+i\epsilon}[\overline u(P_3)(-ie\gamma^\nu)u(P_1)]
	\end{equation}
\end{frame}

\begin{frame}
对初态自旋取平均则有:
\begin{equation}
	\begin{split}
	\frac{d\sigma}{d\Omega}\sim<|\mathcal{M}|^2>&=\frac{e^4}{4(P_1-P_3)^4}[4(P^\mu_1P^\nu_3+P^\mu_3P^\nu_1+(m^2-P_1P_3)g^{\mu\nu})]\\
	&\times[4(P_{2\mu}P_{4\nu}+P_{4\mu}P_{2\nu}+(M^2-P_2P_4)g_{\mu\nu})]\\
	&\equiv\frac{e^4}{q^4}L^{\mu\nu}_{electron}L_{\mu\nu proton}
	\end{split}
\end{equation}

其中m为电子质量,M为质子质量。
\end{frame}

\begin{frame}
通过上述式子可以得到
\begin{block}{核子视作$\frac{1}{2}$自旋点粒子散射截面}
\begin{equation}
	|\frac{d\sigma}{d\Omega}|_{\frac{1}{2}Spin}=|\frac{d\sigma}{d\Omega}|_{Mott}(1+2\tau\tan^2\frac{\theta}{2})
\end{equation}
\end{block}

然而测量发现核子具有反常磁矩,也就意味着核子不是点粒子,他有着内部结构,需要对散射进行进一步的修正。

\begin{equation}
	\begin{split}
	&\mu_p=2.79\mu_N\\
	&\mu_n=-1.91\mu_N
	\end{split}
\end{equation}
\end{frame}

\section{Rosenbluth}
\begin{frame}
\frametitle{考虑核内部结构}
\begin{alertblock}{对上述推导做修正}
$$
L_{\mu\nu proton} \Rightarrow k_{\mu\nu proton}
$$
\end{alertblock}

在顶点处相关项有: $q=P_4-P_2, P_2, P_4, g_{\mu\nu}$。令 $p=P_2$,这样的话p,q相互独立。因此构造$k_{\mu\nu}$的项可以选择有
$$
g_{\mu\nu}, p_\mu p_\nu, q_\mu q_\nu, p_\mu q_\nu+p_\nu q_\mu, p_\mu q_\nu-p_\nu q_\mu
$$

因为$L_{\mu\nu proton}$对$\mu, \nu$对称,因而$k_{\mu\nu proton}$在极限近似下也应该如此。

所以可以排除$p_\mu q_\nu-p_\nu q_\mu$反对称项。


\end{frame}

\begin{frame}
	综上, 有
	\begin{equation}
		k_{\mu\nu proton}= -k_1g_{\mu\nu} +\frac{k_2}{M^2}p_\mu q_\nu + \frac{k_4}{M^2} q_\mu q_\nu + \frac{k_5}{M^2}(p_\mu q_\nu+p_\nu q_\mu)
	\end{equation}

	其中 $k_1, k_2, k_4,k_5$为修正因子, 只与$q^2$相关。

	根据 Ward identity 有$q_\mu k^{\mu\nu}{proton} = 0$, 所以可以得到
	\begin{equation}
		\begin{split}
		&k_4=k_2+\frac{k_1}{q^2}M^2 \\
		&k_5=\frac{k_2}{2}
		\end{split}
	\end{equation}
	进而有
	\begin{equation}
	\begin{split}
	&k_{\mu\nu proton} = k_1(-g_{\mu\nu} + \frac{1}{q^2}q^\mu q^\nu) + \frac{k_2}{M^2}(p_\mu+\frac{q_\mu}{2})(p_\nu+\frac{q_\nu}{2}) \\
	&L^{\mu\nu}_{electron}=[P_1^\mu P_3^\nu+P_3^\mu P_1^\nu+g^{\nu\mu}(m^2-P_1\cdot P_3)]
	\end{split}
	\end{equation}
\end{frame}

\begin{frame}

\begin{equation}
	\frac{d\sigma}{d\Omega} \sim L^{\mu\nu}k_{\mu\nu} = k_1\times(\mathbb{A})+\frac{k_2}{M^2}\times(\mathbb{B})
\end{equation}

同时可以用关系式化简 
\[
\begin{split}
P_1+P_2&=P_3+P_4\\
P_1-P_3&=P_4-P_2=q\\
P_1\cdot q &= P_4 \cdot q = \frac{q^2}{2}\\
P_2\cdot q &= P_3 \cdot q = -\frac{q^2}{2}
\end{split}
\]
化简得到:
\begin{equation}
	\begin{split}
	\mathbb{A} &= 2P_1\cdot P_3 -4m^2\\
	\mathbb{B} &= 2(P_2\cdot p)(P_3 \cdot p) +\frac{q^2}{2}M^2
	\end{split}
\end{equation}
\end{frame}

\begin{frame}

所以综上有:
\begin{equation}
\frac{d\sigma}{d\Omega}\sim 2[k_1((P_1\cdot P_3)-2m^2)+\frac{k_2}{M^2}((P_1\cdot p)(P_3\cdot p))+\frac{q^2M^2}{4}]
\end{equation}

我们有$P_1,P_2,P_3,P_4$的相关信息
\begin{enumerate}
\item P$_1$=(E,0,0,E)
\item P$_2$=(M,0,0,0)
\item P$_3$=(E$^{'}$,0,E$^{'}$sin$\theta$,E$^{'}$cos$\theta$)
\item P$_4$=(E+M-E$^{'}$,0,-E$^{'}$sin$\theta$,E-E$^{'}$cos$\theta$)
\end{enumerate}
带入上式可以得到
\begin{equation}
<|\mathcal M|^2>=\frac{e^4}{4E_1E_3\sin^4\frac{\theta}{2}}[2k_1\sin^2\frac{\theta}{2}+k_2\cos^2\frac{\theta}{2}]
\end{equation}
\end{frame}
\begin{frame}
化简就可以得到 Rosenbluth 形式:
\begin{equation}
\begin{aligned}
\frac{d\sigma}{d\Omega}&=(\frac{\alpha}{4ME_1\sin^2\frac{\theta}{2}})^2\frac{E_3}{E_1}[2k_1(q^2)\sin^2\frac{\theta}{2}+k_2(q^2)\cos^2\frac{\theta}{2}]\\
&=(\frac{\alpha}{4ME_1\sin^2\frac{\theta}{2}})^2\frac{E_3}{E_1}\cos^2\frac{\theta}{2}[k_2(q^2)+2k_1(q^2)\tan^2\frac{\theta}{2}]
\end{aligned}
\end{equation}
通过测量$\frac{d\sigma}{d\Omega}(E,\theta)$可以得到$k_2(q^2)$以及$k_1(q^2)$的具体形式。


但是上式并没有很明显的体现出电和磁形状因子。
\end{frame}
\begin{frame}

因此最开始构造光子和质子的顶点因子时,直接用代表电形状和磁形状因子的F1和F2来构造
\begin{equation}
\begin{split}
&\bar u(P_4)(-ie\Gamma^\mu)u(P_2)\\
&\Gamma^\mu=\gamma^\mu F_1(q^2)+\frac{i\sigma^{\mu\nu}q_\nu}{2M}F_2(q^2)
\end{split}
\end{equation}

重新推导可以得到rosenbluth公式,也即K1和K2由F1和F2表示出来。
\begin{equation}
\frac{d\sigma}{d\Omega}=|\frac{d\sigma}{d\Omega}|_{Mott}\cdot[(F_1^2-\frac{q^2}{4M^2}F_2^2)-\frac{q^2}{2M^2}(F_1+F_2)^2tan^2\frac{\theta}{2}]
\end{equation}
\end{frame}

\end{document}